\chapter{\centering Challenges and Solutions}

During the development of our project, we ran into different challenges as a team and also faced them individually.
In the following section, each member describes their biggest problem(s) and also the solution to it.

% \section{Reduction of boiler plate code}
% \emph{Author Huynh Minh Triet - Matriculation number: 1370690}
% \subsection{Challenge}
% Will building our app following the OOP principle of encapsulation we found out that the majority of 
% the methods in our class are getters and setters, which makes the class extremely long and hard to 
% read.
% \subsection{Solution}
% By using the \emph{Lombok} dependency, all of the getter and setter of the class are automatically 
% generated. Meaning that our team can spend less time on writing repetitive getter, setter code. In addition Lombok allows us to highlight special constructor and getter/setter as these are the only methods that are being shown in the class's code.

\section{Application availability}
\emph{Author Huynh Minh Triet - Matriculation number: 1370690}
\subsection{Challenge}
When first created our app, we did it locally using mySQL hosted locally on Ante's computer. Resulting in the scenario where functions that deal
with the remote database such as: login, sign up or adding events can only be use as long as Ante's personal computer is running. This is
a problem for other team members besides from Ante that want to test with how these database related functionalities work, they are dependent on Ante to get their work done.

\subsection{Solution}
Since one of our team member: Triet Huynh is currently taking the Database course at the same time at the university 
which  supply him a remote database hosted by the university, we decided to use that for our application as a remote 
sever  storing the user's data.

The remote database uses the function sys\_guid() provided by the Oracle database to assign a globally unique ID for
each of the user.This means that no two IDs generated using this method will have the same ID. this function is called
whenever a user has successfully signed up to automatically assign them a unique ID.\\

% Schema of the remote database:
% \begin{lstlisting}[language=SQL]
%  create table UserDB (
%     email    varchar2(100) not null unique,
%     username varchar2(100) not null,
%     hashed_password varchar2(64) not null,
%     user_id raw(16) default sys_guid() constraint 
%     userdb_userid_pk primary key,
%     is_admin number(1) not null,
%     -- This BLOB variable is used to store the local database
%     local_data BLOB 
% )
% \end{lstlisting}


% \section{Multiple connection bug to the remote database}
% \emph{Author Huynh Minh Triet - Matriculation number: 1370690}
% \subsection{Challenge}
% Soon after the usage and deployment of our remote database, we ran into a bug, where due to faulty logical 
% inside the code one instance of the app make continous multiple connection to the remote database. This sudden increase of high volume connection cause the remote database to freeze becoming unresponsive. This results in our application could not be used for a couple of hours. 

% \subsection{Solution}
% We have changed the implementation of the class \emph{DBConnection} that handles the connection to the
% remote database to a singleton class. This makes it impossible for one instance of the app to have multiple 
% connection as a singleton class can only have one object of itself.


\section{Responsiveness and connectivity robustness}
\emph{Author Huynh Minh Triet - Matriculation number: 1370690}

\subsection{Challenge}
At first every interaction with the application that requires stored information is all done through the remote 
database. This cause delay especially when the throughput of the internet connection is slow. Additionally,
the application will come to an abrupt halt if the internet connection is suddenly lost.

Furthermore, the database administrator for the Oracle SQL database that administer the remote server has place a limit 
of one connection to the database for one user account. This problem is discovered when two members of our team tried
to use the application at the same time, wherein they both clicked on the add event button simultaneously causing the 
application to freeze, stopping it from working.

\subsection{Solution}
Though the case mentioned above is relatively rare but we realized that the more each interaction with the application
requires a corresponding query to the remote database the chances of this happening is more often. To solve this
we setup a separate local database using \emph{SQLite} that stores all of the user's data which includes events
and the contact list.

There are three relations that exist on the local database: the event relation, time relation, and participants relation.\\

Additionally a sync button is also added allowing the local database to be uploaded and save in the 
remote database.

\section{Repetitive typing of the participants' emails}
\emph{Author: Huynh Minh Triet - Matriculation number: 1370690: Backend\\
        Co-author    Jorge Vanegas A.- Matriculation number: 1333459: GUI}
\subsection{Challenge}
Earlier version of our application require that each of the participants' emails has to be manually
type in for each events that are being created. While is is feasible for small events with one or two
participants; however, when the number of participants grow to ten or twenty people the chances of typo when 
typing the emails are inevitable. This leads to the reminder not being delivered to the participant and
made the process of adding event a manually intensive and tedious one.

\subsection{Solution}
By adding an additional page to the user's profile page inside setting called  \emph{Contacts}.
Participants can be added into or delete from the contact list store in the local database. 
This contact list is then later shown in the add event page as a scroll panel where the user can 
choose one or multiple participants to add into that event. This allow for frequent participant to be
included with relative ease.


\section{Offline mode}
\emph{Author Huynh Minh Triet - Matriculation number: 1370690}

\subsection{Challenge}
Due to the application not only require the remote database to authenticate the user in login and sign 
up but to also download and upload the local database, the app couldn't be use at all when there
is no internet connection.

\subsection{Solution}
The latest version of the application now has a offline mode which allow the user to use the app 
without any internet connection. When the offline mode is selected and if the application has been
used before, meaning this is not the first time that the user run the application on the current
computer, the application will utilize the current database, allowing them to continue work where
they left off. In the case where this is the first time the application is being run on the machine
a new blank local database will be create.

In addition with the offline button, a new menu bar button called \emph{database}, which when clicked on 
will drop down to reveal two more buttons import and export database; which can be use as backup or to transfer the work onto a different
machine, e.g., you can export the local database to a USB stick and import it later to a different machine.

In the case where internet connection is restored when the user is in offline mode, they can click on the sync 
button which will show the \emph{re-authentication page} where it will ask the user to enter their email and
password to find the account on the remote database. After which they can continue to work in online mode as
usual. The sync button will simply do nothing if internet connection is lost again and the user has to
use the import/export database button to transfer their work. 

\section{Calendar View}
\emph{Author Tim Görß - Matriculation number: 1252200}

\subsection{Challenge}
From the very start of our project we decided against a simple list to display current events of our
users and instead wanted to do a more interact ability and visually pleasing implementation, similar in design to example Google Calendar.


\subsection{Solution}
We use a JTable to display our calendar, where the columns represent the week-days. A Gregorian Calendar object is used to get
the current time of the user and also allow the changing of time. The calendar object also handles leap year problems.
Because each month has a different amount of days and the starting weekdays changes as well, we need more cells than actual days
and create the calendar for each month dynamically.From the first day on wards, the following cells are filled 
with a number according to the days of the month. We use a custom DefaultTableCellRenderer to change the colour of the cells
according to the priority of the events linked to that day. Each cells is also clickable and opens a new frame showing the events
added to that day.

\section{Menu and Side Bar}
\emph{Author Jorge Vanegas A. - Matriculation number: 1333459}

\subsection{Challenge}
As the many features of the application were being added, we found more and more the need to have a place where they could be accessed. For example the Menu which includes Profile, Settings, Exit and Log Out options, as well as Export, Database, and an About page.

\subsection{Solution}
We concurred on the idea to have a fixed side bar, here we could have our sync functions as well as an add event button. Moreover we could show a list of upcoming events to the user. Most importantly however was the implementation of a Menu Bar where all the above mention features could be accessed.

\section{Icons}
\emph{Author Jorge Vanegas A. - Matriculation number: 1333459}

\subsection{Challenge}
GUI with logos and icons would look more pleasing to the user. However we found that Java swing makes it on one side easier to work with the GUI forms, but on the other side a bit more complicated to integrate icons into for example the buttons. Personally could never figure out if it was working with Git or if it was Java swing. We concluded to leave the generic buttons as they are.

\section{Invisible database in JTable}
\emph{Author Ante Maric - Matriculation number: 1273904}

\subsection{Challenge}
While creating the admin GUI that had to display the user database where the admin could edit/delete users from it, the database wouldn't show up in the GUI (empty JTable - invisible database) even though the database was implemented correctly in the code (Tested with System.out.println(); functions).


\subsection{Solution}
After some try and fail options, thoughts about the code being wrong, the solution to it was to manually add the columns and name them above the main section of the actual code to make them visible in the JTable. The database was finally displayed properly and the users could be selected for further function implementation.

\section{Database not updating changes}
\emph{Author Ante Maric - Matriculation number: 1273904}

\subsection{Challenge}
After the admin GUI was created and the database fully displayed and ready to be worked on and after successfully implementing the delete function, we started working on implementing the edit function where the admin can select a user from the database and change his username and/or email. After coding for a while and testing out the implemented SQL queries ("UPDATE XYZ FROM ABC WHERE 123"), the database should be updating the users according to our changes made from the GUI options but it did not. We didn't receive any errors on the execution of the program and the function seemed to work in theory, but as shown in the database it didn't change anything.


\subsection{Solution}
Days have passed and we tried several different changes/options but we didn't get any good results. After additional changes, we received an SQL exception (unique constraint) where it showed us that some attributes in the database where set up to not allow any duplicates with the changes so we had to change them in order to allow the manual edit from our admin panel. As we fixed the database problems, we had to also make some changes to the SQL Queries and combine them with some get methods from our edit function and everything finally updated on the database when we pressed the apply changes button.

\section{Emails not being received and accounts getting banned}
\emph{Author Ante Maric - Matriculation number: 1273904}

\subsection{Challenge}
While testing out the email API, we created some dummy accounts on gmail and tried to send out test emails to see if they were properly sent out and received by our account. The emails were sent out from our program, but there were no emails in the mailbox. 
Email accounts couldn't be used anymore for testing as they were getting banned.


\subsection{Solution}
After some research, we found out that you have to turn ON the "Allow less secure apps to access your account" in the gmail settings to receive the emails on our gmail account. Afterwards, everything worked as it should. After fixing the problem with the receiving emails, the accounts were getting target banned after a few days as they received some test emails and they have been marked as unsafe from gmail as we don't have any licences as other apps/programs have to prove that we are not malicious (problem can't be solved).