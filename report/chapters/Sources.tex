\chapter{\centering Sources}

% In this chapter, we will provide information on where we got the code from that might have been used in our project and overall information where we got our ideas/tutorials from.

\section{\emph{Ante Maric - Matriculation number: 1273904}}

\subsection{Email API}
% Good in-depth explanation on how to send emails and how the java mail works overall, a site [1] and a YouTube video tutorial [2]. \newline

\emph{[1]link:} https://www.javatpoint.com/example-of-sending-email-using-java-mail-api 

\emph{[2]link:} https://www.youtube.com/watch?v=A7HAB5whD6I
\newline

\emph{[1]Author:} © Copyright 2011-2021 www.javatpoint.com. All rights reserved. Developed by JavaTpoint.

\emph{[2]Author:} Genuine Coder

\emph{Last accessed:} on the 11th of February, 2022 - 8.11 PM

\subsection{Reminder functionality}
% [1]High quality information on how the Java time and timerTask libraries work and on how to implement them\newline
[2] Youtube tutorial about timers and timerTasks well exlplained\newline

[1]\emph{link:} https://www.baeldung.com/java-timer-and-timertask 

\emph{Author:} Eugen Paraschiv - Baeldung\newline

[2]\emph{link:} https://youtu.be/QEF62Fm81h4 

\emph{Author:} Bro Code\newline

\emph{Last accessed:} on the 11th of February, 2022 - 8.26 PM

\section{\emph{Huynh Minh Triet - Matriculation number: 1370690}}

\subsection{Jfilechooser file type filter}

\emph{link:} https://www.youtube.com/watch?v=lFkYt2jKrYc \newline

% The entire implementation of the class FileTypeFilter.java is copied from the previous youtube video.
% The class allow for the Jfilechooser to only display a certain file type when choosing a file.

\section{\emph{Jorge Vanegas A. - Matriculation number: 1333459}}
\subsection{Menu Bar}
https://www.youtube.com/watch?v=dwLkDGm5EBc \\
\emph{Last accessed:} on the 26th of January, 2022
% https://www.geeksforgeeks.org/java-swing-jmenubar/ \newline
% \par Picked up a few ideas and code from these two links for the JMenuBar and learned how to implement the multiple menus options into our software.
